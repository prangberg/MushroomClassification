\documentclass[]{article}
\usepackage{lmodern}
\usepackage{amssymb,amsmath}
\usepackage{ifxetex,ifluatex}
\usepackage{fixltx2e} % provides \textsubscript
\ifnum 0\ifxetex 1\fi\ifluatex 1\fi=0 % if pdftex
  \usepackage[T1]{fontenc}
  \usepackage[utf8]{inputenc}
\else % if luatex or xelatex
  \ifxetex
    \usepackage{mathspec}
  \else
    \usepackage{fontspec}
  \fi
  \defaultfontfeatures{Ligatures=TeX,Scale=MatchLowercase}
\fi
% use upquote if available, for straight quotes in verbatim environments
\IfFileExists{upquote.sty}{\usepackage{upquote}}{}
% use microtype if available
\IfFileExists{microtype.sty}{%
\usepackage{microtype}
\UseMicrotypeSet[protrusion]{basicmath} % disable protrusion for tt fonts
}{}
\usepackage[margin=1in]{geometry}
\usepackage{hyperref}
\hypersetup{unicode=true,
            pdftitle={HarvardX: PH125.9x Professional Certificate in Data Science - Choose Your Own Project},
            pdfauthor={Stefan Prangenberg},
            pdfborder={0 0 0},
            breaklinks=true}
\urlstyle{same}  % don't use monospace font for urls
\usepackage{color}
\usepackage{fancyvrb}
\newcommand{\VerbBar}{|}
\newcommand{\VERB}{\Verb[commandchars=\\\{\}]}
\DefineVerbatimEnvironment{Highlighting}{Verbatim}{commandchars=\\\{\}}
% Add ',fontsize=\small' for more characters per line
\usepackage{framed}
\definecolor{shadecolor}{RGB}{248,248,248}
\newenvironment{Shaded}{\begin{snugshade}}{\end{snugshade}}
\newcommand{\KeywordTok}[1]{\textcolor[rgb]{0.13,0.29,0.53}{\textbf{#1}}}
\newcommand{\DataTypeTok}[1]{\textcolor[rgb]{0.13,0.29,0.53}{#1}}
\newcommand{\DecValTok}[1]{\textcolor[rgb]{0.00,0.00,0.81}{#1}}
\newcommand{\BaseNTok}[1]{\textcolor[rgb]{0.00,0.00,0.81}{#1}}
\newcommand{\FloatTok}[1]{\textcolor[rgb]{0.00,0.00,0.81}{#1}}
\newcommand{\ConstantTok}[1]{\textcolor[rgb]{0.00,0.00,0.00}{#1}}
\newcommand{\CharTok}[1]{\textcolor[rgb]{0.31,0.60,0.02}{#1}}
\newcommand{\SpecialCharTok}[1]{\textcolor[rgb]{0.00,0.00,0.00}{#1}}
\newcommand{\StringTok}[1]{\textcolor[rgb]{0.31,0.60,0.02}{#1}}
\newcommand{\VerbatimStringTok}[1]{\textcolor[rgb]{0.31,0.60,0.02}{#1}}
\newcommand{\SpecialStringTok}[1]{\textcolor[rgb]{0.31,0.60,0.02}{#1}}
\newcommand{\ImportTok}[1]{#1}
\newcommand{\CommentTok}[1]{\textcolor[rgb]{0.56,0.35,0.01}{\textit{#1}}}
\newcommand{\DocumentationTok}[1]{\textcolor[rgb]{0.56,0.35,0.01}{\textbf{\textit{#1}}}}
\newcommand{\AnnotationTok}[1]{\textcolor[rgb]{0.56,0.35,0.01}{\textbf{\textit{#1}}}}
\newcommand{\CommentVarTok}[1]{\textcolor[rgb]{0.56,0.35,0.01}{\textbf{\textit{#1}}}}
\newcommand{\OtherTok}[1]{\textcolor[rgb]{0.56,0.35,0.01}{#1}}
\newcommand{\FunctionTok}[1]{\textcolor[rgb]{0.00,0.00,0.00}{#1}}
\newcommand{\VariableTok}[1]{\textcolor[rgb]{0.00,0.00,0.00}{#1}}
\newcommand{\ControlFlowTok}[1]{\textcolor[rgb]{0.13,0.29,0.53}{\textbf{#1}}}
\newcommand{\OperatorTok}[1]{\textcolor[rgb]{0.81,0.36,0.00}{\textbf{#1}}}
\newcommand{\BuiltInTok}[1]{#1}
\newcommand{\ExtensionTok}[1]{#1}
\newcommand{\PreprocessorTok}[1]{\textcolor[rgb]{0.56,0.35,0.01}{\textit{#1}}}
\newcommand{\AttributeTok}[1]{\textcolor[rgb]{0.77,0.63,0.00}{#1}}
\newcommand{\RegionMarkerTok}[1]{#1}
\newcommand{\InformationTok}[1]{\textcolor[rgb]{0.56,0.35,0.01}{\textbf{\textit{#1}}}}
\newcommand{\WarningTok}[1]{\textcolor[rgb]{0.56,0.35,0.01}{\textbf{\textit{#1}}}}
\newcommand{\AlertTok}[1]{\textcolor[rgb]{0.94,0.16,0.16}{#1}}
\newcommand{\ErrorTok}[1]{\textcolor[rgb]{0.64,0.00,0.00}{\textbf{#1}}}
\newcommand{\NormalTok}[1]{#1}
\usepackage{graphicx,grffile}
\makeatletter
\def\maxwidth{\ifdim\Gin@nat@width>\linewidth\linewidth\else\Gin@nat@width\fi}
\def\maxheight{\ifdim\Gin@nat@height>\textheight\textheight\else\Gin@nat@height\fi}
\makeatother
% Scale images if necessary, so that they will not overflow the page
% margins by default, and it is still possible to overwrite the defaults
% using explicit options in \includegraphics[width, height, ...]{}
\setkeys{Gin}{width=\maxwidth,height=\maxheight,keepaspectratio}
\IfFileExists{parskip.sty}{%
\usepackage{parskip}
}{% else
\setlength{\parindent}{0pt}
\setlength{\parskip}{6pt plus 2pt minus 1pt}
}
\setlength{\emergencystretch}{3em}  % prevent overfull lines
\providecommand{\tightlist}{%
  \setlength{\itemsep}{0pt}\setlength{\parskip}{0pt}}
\setcounter{secnumdepth}{0}
% Redefines (sub)paragraphs to behave more like sections
\ifx\paragraph\undefined\else
\let\oldparagraph\paragraph
\renewcommand{\paragraph}[1]{\oldparagraph{#1}\mbox{}}
\fi
\ifx\subparagraph\undefined\else
\let\oldsubparagraph\subparagraph
\renewcommand{\subparagraph}[1]{\oldsubparagraph{#1}\mbox{}}
\fi

%%% Use protect on footnotes to avoid problems with footnotes in titles
\let\rmarkdownfootnote\footnote%
\def\footnote{\protect\rmarkdownfootnote}

%%% Change title format to be more compact
\usepackage{titling}

% Create subtitle command for use in maketitle
\newcommand{\subtitle}[1]{
  \posttitle{
    \begin{center}\large#1\end{center}
    }
}

\setlength{\droptitle}{-2em}

  \title{HarvardX: PH125.9x Professional Certificate in Data Science - Choose
Your Own Project}
    \pretitle{\vspace{\droptitle}\centering\huge}
  \posttitle{\par}
    \author{Stefan Prangenberg}
    \preauthor{\centering\large\emph}
  \postauthor{\par}
      \predate{\centering\large\emph}
  \postdate{\par}
    \date{6/15/2019}


\begin{document}
\maketitle

*This project and the source core are available online at
\url{https://github.com/prangberg/MushroomClassification}

This report is part of the final course in the HarvardX Professional
Certificate in Data Science applying machine learning techniques that go
beyond standard linear regression. * \#Introduction

The goal of this project is to predict if mushrooms are edible or
poisonous - based on their numerous attributes such as cap color
ornumber of rings. Since we want to avoid eating a poisonous mushroom,
we strive for 100\% accuracy.

I downloaded the data from Kaggle
\url{https://www.kaggle.com/uciml/mushroom-classification} and saved it
as a .csv in a local folder.

This dataset includes descriptions of hypothetical samples corresponding
to 23 species of gilled mushrooms in the Agaricus and Lepiota Family
Mushroom drawn from The Audubon Society Field Guide to North American
Mushrooms (1981). Each species is identified as definitely edible,
definitely poisonous, or of unknown edibility and not recommended. This
latter class was combined with the poisonous one. The Guide clearly
states that there is no simple rule for determining the edibility of a
mushroom; no rule like ``leaflets three, let it be'' for Poisonous Oak
and Ivy.

\section{Analysis}\label{analysis}

We use the folloring libraries for our analysis:

\begin{Shaded}
\begin{Highlighting}[]
\KeywordTok{library}\NormalTok{(tidyverse)}
\KeywordTok{library}\NormalTok{(caret)}
\KeywordTok{library}\NormalTok{(randomForest)}
\KeywordTok{library}\NormalTok{(ggplot2)}
\end{Highlighting}
\end{Shaded}

First, we ingest the CSV

\begin{Shaded}
\begin{Highlighting}[]
\NormalTok{mushrooms <-}\StringTok{ }\KeywordTok{read.csv}\NormalTok{(}\StringTok{"mushrooms.csv"}\NormalTok{, }\DataTypeTok{colClasses =} \StringTok{"character"}\NormalTok{) }
\end{Highlighting}
\end{Shaded}

Note: The method read.csv worked better for me than read\_csv, since the
later tried to convert `bruises' and `gill-attachment' to a logical
variable, which caused problems.

\begin{Shaded}
\begin{Highlighting}[]
\KeywordTok{dim}\NormalTok{(mushrooms)}
\end{Highlighting}
\end{Shaded}

\begin{verbatim}
## [1] 8124   23
\end{verbatim}

The dataset has 8124 entries with 23 columns.

\begin{Shaded}
\begin{Highlighting}[]
\KeywordTok{glimpse}\NormalTok{(mushrooms)}
\end{Highlighting}
\end{Shaded}

\begin{verbatim}
## Observations: 8,124
## Variables: 23
## $ class                    <chr> "p", "e", "e", "p", "e", "e", "e", "e...
## $ cap.shape                <chr> "x", "x", "b", "x", "x", "x", "b", "b...
## $ cap.surface              <chr> "s", "s", "s", "y", "s", "y", "s", "y...
## $ cap.color                <chr> "n", "y", "w", "w", "g", "y", "w", "w...
## $ bruises                  <chr> "t", "t", "t", "t", "f", "t", "t", "t...
## $ odor                     <chr> "p", "a", "l", "p", "n", "a", "a", "l...
## $ gill.attachment          <chr> "f", "f", "f", "f", "f", "f", "f", "f...
## $ gill.spacing             <chr> "c", "c", "c", "c", "w", "c", "c", "c...
## $ gill.size                <chr> "n", "b", "b", "n", "b", "b", "b", "b...
## $ gill.color               <chr> "k", "k", "n", "n", "k", "n", "g", "n...
## $ stalk.shape              <chr> "e", "e", "e", "e", "t", "e", "e", "e...
## $ stalk.root               <chr> "e", "c", "c", "e", "e", "c", "c", "c...
## $ stalk.surface.above.ring <chr> "s", "s", "s", "s", "s", "s", "s", "s...
## $ stalk.surface.below.ring <chr> "s", "s", "s", "s", "s", "s", "s", "s...
## $ stalk.color.above.ring   <chr> "w", "w", "w", "w", "w", "w", "w", "w...
## $ stalk.color.below.ring   <chr> "w", "w", "w", "w", "w", "w", "w", "w...
## $ veil.type                <chr> "p", "p", "p", "p", "p", "p", "p", "p...
## $ veil.color               <chr> "w", "w", "w", "w", "w", "w", "w", "w...
## $ ring.number              <chr> "o", "o", "o", "o", "o", "o", "o", "o...
## $ ring.type                <chr> "p", "p", "p", "p", "e", "p", "p", "p...
## $ spore.print.color        <chr> "k", "n", "n", "k", "n", "k", "k", "n...
## $ population               <chr> "s", "n", "n", "s", "a", "n", "n", "s...
## $ habitat                  <chr> "u", "g", "m", "u", "g", "g", "m", "m...
\end{verbatim}

The values are single letter. We can find the meaning in the definition
on kaggle.com: The columns are already labeled, but the single letter
abbreviations are not very meaningful.

Definition of columns from kaggle

\begin{itemize}
\tightlist
\item
  classes: edible=e, poisonous=p
\item
  cap-shape: bell=b,conical=c,convex=x,flat=f, knobbed=k,sunken=s
\item
  cap-surface: fibrous=f,grooves=g,scaly=y,smooth=s
\item
  cap-color:
  brown=n,buff=b,cinnamon=c,gray=g,green=r,pink=p,purple=u,red=e,white=w,yellow=y
\item
  bruises: bruises=t,no=f
\item
  odor:
  almond=a,anise=l,creosote=c,fishy=y,foul=f,musty=m,none=n,pungent=p,spicy=s
\item
  gill-attachment: attached=a,descending=d,free=f,notched=n
\item
  gill-spacing: close=c,crowded=w,distant=d
\item
  gill-size: broad=b,narrow=n
\item
  gill-color:
  black=k,brown=n,buff=b,chocolate=h,gray=g,green=r,orange=o,pink=p,purple=u,red=e,white=w,yellow=y
\item
  stalk-shape: enlarging=e,tapering=t
\item
  stalk-root:
  bulbous=b,club=c,cup=u,equal=e,rhizomorphs=z,rooted=r,missing=?
\item
  stalk-surface-above-ring: fibrous=f,scaly=y,silky=k,smooth=s
\item
  stalk-surface-below-ring: fibrous=f,scaly=y,silky=k,smooth=s
\item
  stalk-color-above-ring:
  brown=n,buff=b,cinnamon=c,gray=g,orange=o,pink=p,red=e,white=w,yellow=y
\item
  stalk-color-below-ring:
  brown=n,buff=b,cinnamon=c,gray=g,orange=o,pink=p,red=e,white=w,yellow=y
\item
  veil-type: partial=p,universal=u
\item
  veil-color: brown=n,orange=o,white=w,yellow=y
\item
  ring-number: none=n,one=o,two=t
\item
  ring-type:
  cobwebby=c,evanescent=e,flaring=f,large=l,none=n,pendant=p,sheathing=s,zone=z
\item
  spore-print-color:
  black=k,brown=n,buff=b,chocolate=h,green=r,orange=o,purple=u,white=w,yellow=y
\item
  population:
  abundant=a,clustered=c,numerous=n,scattered=s,several=v,solitary=y
\item
  habitat: grasses=g,leaves=l,meadows=m,paths=p,urban=u,waste=w,woods=d
\end{itemize}

\subsection{Data Cleaning}\label{data-cleaning}

Labeling the columns: We'll create human-friendly names for each
category. First we map the data to a factor:

\begin{Shaded}
\begin{Highlighting}[]
\NormalTok{mushrooms <-}\StringTok{ }\NormalTok{mushrooms }\OperatorTok\StringTok{ }\KeywordTok{map_df}\NormalTok{(}\ControlFlowTok{function}\NormalTok{(.x) }\KeywordTok{as.factor}\NormalTok{(.x))}
\KeywordTok{str}\NormalTok{(mushrooms)}
\end{Highlighting}
\end{Shaded}

\begin{verbatim}
## Classes 'tbl_df', 'tbl' and 'data.frame':    8124 obs. of  23 variables:
##  $ class                   : Factor w/ 2 levels "e","p": 2 1 1 2 1 1 1 1 2 1 ...
##  $ cap.shape               : Factor w/ 6 levels "b","c","f","k",..: 6 6 1 6 6 6 1 1 6 1 ...
##  $ cap.surface             : Factor w/ 4 levels "f","g","s","y": 3 3 3 4 3 4 3 4 4 3 ...
##  $ cap.color               : Factor w/ 10 levels "b","c","e","g",..: 5 10 9 9 4 10 9 9 9 10 ...
##  $ bruises                 : Factor w/ 2 levels "f","t": 2 2 2 2 1 2 2 2 2 2 ...
##  $ odor                    : Factor w/ 9 levels "a","c","f","l",..: 7 1 4 7 6 1 1 4 7 1 ...
##  $ gill.attachment         : Factor w/ 2 levels "a","f": 2 2 2 2 2 2 2 2 2 2 ...
##  $ gill.spacing            : Factor w/ 2 levels "c","w": 1 1 1 1 2 1 1 1 1 1 ...
##  $ gill.size               : Factor w/ 2 levels "b","n": 2 1 1 2 1 1 1 1 2 1 ...
##  $ gill.color              : Factor w/ 12 levels "b","e","g","h",..: 5 5 6 6 5 6 3 6 8 3 ...
##  $ stalk.shape             : Factor w/ 2 levels "e","t": 1 1 1 1 2 1 1 1 1 1 ...
##  $ stalk.root              : Factor w/ 5 levels "?","b","c","e",..: 4 3 3 4 4 3 3 3 4 3 ...
##  $ stalk.surface.above.ring: Factor w/ 4 levels "f","k","s","y": 3 3 3 3 3 3 3 3 3 3 ...
##  $ stalk.surface.below.ring: Factor w/ 4 levels "f","k","s","y": 3 3 3 3 3 3 3 3 3 3 ...
##  $ stalk.color.above.ring  : Factor w/ 9 levels "b","c","e","g",..: 8 8 8 8 8 8 8 8 8 8 ...
##  $ stalk.color.below.ring  : Factor w/ 9 levels "b","c","e","g",..: 8 8 8 8 8 8 8 8 8 8 ...
##  $ veil.type               : Factor w/ 1 level "p": 1 1 1 1 1 1 1 1 1 1 ...
##  $ veil.color              : Factor w/ 4 levels "n","o","w","y": 3 3 3 3 3 3 3 3 3 3 ...
##  $ ring.number             : Factor w/ 3 levels "n","o","t": 2 2 2 2 2 2 2 2 2 2 ...
##  $ ring.type               : Factor w/ 5 levels "e","f","l","n",..: 5 5 5 5 1 5 5 5 5 5 ...
##  $ spore.print.color       : Factor w/ 9 levels "b","h","k","n",..: 3 4 4 3 4 3 3 4 3 3 ...
##  $ population              : Factor w/ 6 levels "a","c","n","s",..: 4 3 3 4 1 3 3 4 5 4 ...
##  $ habitat                 : Factor w/ 7 levels "d","g","l","m",..: 6 2 4 6 2 2 4 4 2 4 ...
\end{verbatim}

Notice that ``veil.type'' only has one single level, so this variable is
fairly useless for any analysis and we can ignore it. In a larger
dataset it would make sense to remove it in order to reduce complexity
and improve performance, but in this small dataset I omit this step.
Every other variabel has 2-12 different levels.

For each variable we define meaningful (and easy to understand) variable
values, baszed on the definitions on Kaggle.

\begin{Shaded}
\begin{Highlighting}[]
\KeywordTok{levels}\NormalTok{(mushrooms}\OperatorTok{$}\NormalTok{class) <-}\StringTok{ }\KeywordTok{c}\NormalTok{(}\StringTok{"edible"}\NormalTok{, }\StringTok{"poisonous"}\NormalTok{)}
\KeywordTok{levels}\NormalTok{(mushrooms}\OperatorTok{$}\NormalTok{cap.shape) <-}\StringTok{ }\KeywordTok{c}\NormalTok{(}\StringTok{"bell"}\NormalTok{, }\StringTok{"conical"}\NormalTok{, }\StringTok{"flat"}\NormalTok{, }\StringTok{"knobbed"}\NormalTok{, }\StringTok{"sunken"}\NormalTok{, }\StringTok{"convex"}\NormalTok{)}
\KeywordTok{levels}\NormalTok{(mushrooms}\OperatorTok{$}\NormalTok{cap.color) <-}\StringTok{ }\KeywordTok{c}\NormalTok{(}\StringTok{"buff"}\NormalTok{, }\StringTok{"cinnamon"}\NormalTok{, }\StringTok{"red"}\NormalTok{, }\StringTok{"gray"}\NormalTok{, }\StringTok{"brown"}\NormalTok{, }\StringTok{"pink"}\NormalTok{, }\StringTok{"green"}\NormalTok{, }\StringTok{"purple"}\NormalTok{, }\StringTok{"white"}\NormalTok{, }\StringTok{"yellow"}\NormalTok{)}
\KeywordTok{levels}\NormalTok{(mushrooms}\OperatorTok{$}\NormalTok{cap.surface) <-}\StringTok{ }\KeywordTok{c}\NormalTok{(}\StringTok{"fibrous"}\NormalTok{, }\StringTok{"grooves"}\NormalTok{, }\StringTok{"scaly"}\NormalTok{, }\StringTok{"smooth"}\NormalTok{)}
\KeywordTok{levels}\NormalTok{(mushrooms}\OperatorTok{$}\NormalTok{bruises) <-}\StringTok{ }\KeywordTok{c}\NormalTok{(}\StringTok{"no"}\NormalTok{, }\StringTok{"yes"}\NormalTok{)}
\KeywordTok{levels}\NormalTok{(mushrooms}\OperatorTok{$}\NormalTok{odor) <-}\StringTok{ }\KeywordTok{c}\NormalTok{(}\StringTok{"almond"}\NormalTok{, }\StringTok{"creosote"}\NormalTok{, }\StringTok{"foul"}\NormalTok{, }\StringTok{"anise"}\NormalTok{, }\StringTok{"musty"}\NormalTok{, }\StringTok{"none"}\NormalTok{, }\StringTok{"pungent"}\NormalTok{, }\StringTok{"spicy"}\NormalTok{, }\StringTok{"fishy"}\NormalTok{)}
\KeywordTok{levels}\NormalTok{(mushrooms}\OperatorTok{$}\NormalTok{gill.attachment) <-}\StringTok{ }\KeywordTok{c}\NormalTok{(}\StringTok{"attached"}\NormalTok{, }\StringTok{"free"}\NormalTok{)}
\KeywordTok{levels}\NormalTok{(mushrooms}\OperatorTok{$}\NormalTok{gill.spacing) <-}\StringTok{ }\KeywordTok{c}\NormalTok{(}\StringTok{"close"}\NormalTok{, }\StringTok{"crowded"}\NormalTok{)}
\KeywordTok{levels}\NormalTok{(mushrooms}\OperatorTok{$}\NormalTok{gill.size) <-}\StringTok{ }\KeywordTok{c}\NormalTok{(}\StringTok{"broad"}\NormalTok{, }\StringTok{"narrow"}\NormalTok{)}
\KeywordTok{levels}\NormalTok{(mushrooms}\OperatorTok{$}\NormalTok{gill.color) <-}\StringTok{ }\KeywordTok{c}\NormalTok{(}\StringTok{"buff"}\NormalTok{, }\StringTok{"red"}\NormalTok{, }\StringTok{"gray"}\NormalTok{, }\StringTok{"chocolate"}\NormalTok{, }\StringTok{"black"}\NormalTok{, }\StringTok{"brown"}\NormalTok{, }\StringTok{"orange"}\NormalTok{, }\StringTok{"pink"}\NormalTok{, }\StringTok{"green"}\NormalTok{, }\StringTok{"purple"}\NormalTok{, }\StringTok{"white"}\NormalTok{, }\StringTok{"yellow"}\NormalTok{)}
\KeywordTok{levels}\NormalTok{(mushrooms}\OperatorTok{$}\NormalTok{stalk.shape) <-}\StringTok{ }\KeywordTok{c}\NormalTok{(}\StringTok{"enlarging"}\NormalTok{, }\StringTok{"tapering"}\NormalTok{)}
\KeywordTok{levels}\NormalTok{(mushrooms}\OperatorTok{$}\NormalTok{stalk.root) <-}\StringTok{ }\KeywordTok{c}\NormalTok{(}\StringTok{"missing"}\NormalTok{, }\StringTok{"bulbous"}\NormalTok{, }\StringTok{"club"}\NormalTok{, }\StringTok{"equal"}\NormalTok{, }\StringTok{"rooted"}\NormalTok{)}
\KeywordTok{levels}\NormalTok{(mushrooms}\OperatorTok{$}\NormalTok{stalk.surface.above.ring) <-}\StringTok{ }\KeywordTok{c}\NormalTok{(}\StringTok{"fibrous"}\NormalTok{, }\StringTok{"silky"}\NormalTok{, }\StringTok{"smooth"}\NormalTok{, }\StringTok{"scaly"}\NormalTok{)}
\KeywordTok{levels}\NormalTok{(mushrooms}\OperatorTok{$}\NormalTok{stalk.surface.below.ring) <-}\StringTok{ }\KeywordTok{c}\NormalTok{(}\StringTok{"fibrous"}\NormalTok{, }\StringTok{"silky"}\NormalTok{, }\StringTok{"smooth"}\NormalTok{, }\StringTok{"scaly"}\NormalTok{)}
\KeywordTok{levels}\NormalTok{(mushrooms}\OperatorTok{$}\NormalTok{stalk.color.above.ring) <-}\StringTok{ }\KeywordTok{c}\NormalTok{(}\StringTok{"buff"}\NormalTok{, }\StringTok{"cinnamon"}\NormalTok{, }\StringTok{"red"}\NormalTok{, }\StringTok{"gray"}\NormalTok{, }\StringTok{"brown"}\NormalTok{, }\StringTok{"pink"}\NormalTok{, }\StringTok{"green"}\NormalTok{, }\StringTok{"purple"}\NormalTok{, }\StringTok{"white"}\NormalTok{, }\StringTok{"yellow"}\NormalTok{)}
\KeywordTok{levels}\NormalTok{(mushrooms}\OperatorTok{$}\NormalTok{stalk.color.below.ring) <-}\StringTok{ }\KeywordTok{c}\NormalTok{(}\StringTok{"buff"}\NormalTok{, }\StringTok{"cinnamon"}\NormalTok{, }\StringTok{"red"}\NormalTok{, }\StringTok{"gray"}\NormalTok{, }\StringTok{"brown"}\NormalTok{, }\StringTok{"pink"}\NormalTok{, }\StringTok{"green"}\NormalTok{, }\StringTok{"purple"}\NormalTok{, }\StringTok{"white"}\NormalTok{, }\StringTok{"yellow"}\NormalTok{)}
\KeywordTok{levels}\NormalTok{(mushrooms}\OperatorTok{$}\NormalTok{veil.type) <-}\StringTok{ "partial"}
\KeywordTok{levels}\NormalTok{(mushrooms}\OperatorTok{$}\NormalTok{veil.color) <-}\StringTok{ }\KeywordTok{c}\NormalTok{(}\StringTok{"brown"}\NormalTok{, }\StringTok{"orange"}\NormalTok{, }\StringTok{"white"}\NormalTok{, }\StringTok{"yellow"}\NormalTok{)}
\KeywordTok{levels}\NormalTok{(mushrooms}\OperatorTok{$}\NormalTok{ring.number) <-}\StringTok{ }\KeywordTok{c}\NormalTok{(}\StringTok{"none"}\NormalTok{, }\StringTok{"one"}\NormalTok{, }\StringTok{"two"}\NormalTok{)}
\KeywordTok{levels}\NormalTok{(mushrooms}\OperatorTok{$}\NormalTok{ring.type) <-}\StringTok{ }\KeywordTok{c}\NormalTok{(}\StringTok{"evanescent"}\NormalTok{, }\StringTok{"flaring"}\NormalTok{, }\StringTok{"large"}\NormalTok{, }\StringTok{"none"}\NormalTok{, }\StringTok{"pendant"}\NormalTok{)}
\KeywordTok{levels}\NormalTok{(mushrooms}\OperatorTok{$}\NormalTok{spore.print.color) <-}\StringTok{ }\KeywordTok{c}\NormalTok{(}\StringTok{"buff"}\NormalTok{, }\StringTok{"chocolate"}\NormalTok{, }\StringTok{"black"}\NormalTok{, }\StringTok{"brown"}\NormalTok{, }\StringTok{"orange"}\NormalTok{, }\StringTok{"green"}\NormalTok{, }\StringTok{"purple"}\NormalTok{, }\StringTok{"white"}\NormalTok{, }\StringTok{"yellow"}\NormalTok{)}
\KeywordTok{levels}\NormalTok{(mushrooms}\OperatorTok{$}\NormalTok{population) <-}\StringTok{ }\KeywordTok{c}\NormalTok{(}\StringTok{"abundant"}\NormalTok{, }\StringTok{"clustered"}\NormalTok{, }\StringTok{"numerous"}\NormalTok{, }\StringTok{"scattered"}\NormalTok{, }\StringTok{"several"}\NormalTok{, }\StringTok{"solitary"}\NormalTok{)}
\KeywordTok{levels}\NormalTok{(mushrooms}\OperatorTok{$}\NormalTok{habitat) <-}\StringTok{ }\KeywordTok{c}\NormalTok{(}\StringTok{"wood"}\NormalTok{, }\StringTok{"grasses"}\NormalTok{, }\StringTok{"leaves"}\NormalTok{, }\StringTok{"meadows"}\NormalTok{, }\StringTok{"paths"}\NormalTok{, }\StringTok{"urban"}\NormalTok{, }\StringTok{"waste"}\NormalTok{)}

\KeywordTok{str}\NormalTok{(mushrooms)}
\end{Highlighting}
\end{Shaded}

\begin{verbatim}
## Classes 'tbl_df', 'tbl' and 'data.frame':    8124 obs. of  23 variables:
##  $ class                   : Factor w/ 2 levels "edible","poisonous": 2 1 1 2 1 1 1 1 2 1 ...
##  $ cap.shape               : Factor w/ 6 levels "bell","conical",..: 6 6 1 6 6 6 1 1 6 1 ...
##  $ cap.surface             : Factor w/ 4 levels "fibrous","grooves",..: 3 3 3 4 3 4 3 4 4 3 ...
##  $ cap.color               : Factor w/ 10 levels "buff","cinnamon",..: 5 10 9 9 4 10 9 9 9 10 ...
##  $ bruises                 : Factor w/ 2 levels "no","yes": 2 2 2 2 1 2 2 2 2 2 ...
##  $ odor                    : Factor w/ 9 levels "almond","creosote",..: 7 1 4 7 6 1 1 4 7 1 ...
##  $ gill.attachment         : Factor w/ 2 levels "attached","free": 2 2 2 2 2 2 2 2 2 2 ...
##  $ gill.spacing            : Factor w/ 2 levels "close","crowded": 1 1 1 1 2 1 1 1 1 1 ...
##  $ gill.size               : Factor w/ 2 levels "broad","narrow": 2 1 1 2 1 1 1 1 2 1 ...
##  $ gill.color              : Factor w/ 12 levels "buff","red","gray",..: 5 5 6 6 5 6 3 6 8 3 ...
##  $ stalk.shape             : Factor w/ 2 levels "enlarging","tapering": 1 1 1 1 2 1 1 1 1 1 ...
##  $ stalk.root              : Factor w/ 5 levels "missing","bulbous",..: 4 3 3 4 4 3 3 3 4 3 ...
##  $ stalk.surface.above.ring: Factor w/ 4 levels "fibrous","silky",..: 3 3 3 3 3 3 3 3 3 3 ...
##  $ stalk.surface.below.ring: Factor w/ 4 levels "fibrous","silky",..: 3 3 3 3 3 3 3 3 3 3 ...
##  $ stalk.color.above.ring  : Factor w/ 10 levels "buff","cinnamon",..: 8 8 8 8 8 8 8 8 8 8 ...
##  $ stalk.color.below.ring  : Factor w/ 10 levels "buff","cinnamon",..: 8 8 8 8 8 8 8 8 8 8 ...
##  $ veil.type               : Factor w/ 1 level "partial": 1 1 1 1 1 1 1 1 1 1 ...
##  $ veil.color              : Factor w/ 4 levels "brown","orange",..: 3 3 3 3 3 3 3 3 3 3 ...
##  $ ring.number             : Factor w/ 3 levels "none","one","two": 2 2 2 2 2 2 2 2 2 2 ...
##  $ ring.type               : Factor w/ 5 levels "evanescent","flaring",..: 5 5 5 5 1 5 5 5 5 5 ...
##  $ spore.print.color       : Factor w/ 9 levels "buff","chocolate",..: 3 4 4 3 4 3 3 4 3 3 ...
##  $ population              : Factor w/ 6 levels "abundant","clustered",..: 4 3 3 4 1 3 3 4 5 4 ...
##  $ habitat                 : Factor w/ 7 levels "wood","grasses",..: 6 2 4 6 2 2 4 4 2 4 ...
\end{verbatim}

The data is looking good - we're ready to explore it

\subsection{Data Exploration and
Visualization}\label{data-exploration-and-visualization}

The most important information for the person finding a mushroom is
weather it is ebible or poisonous.

\begin{verbatim}
## [1] 8124   23
\end{verbatim}

\begin{verbatim}
##           x freq
## 1    edible 4208
## 2 poisonous 3916
\end{verbatim}

Almost half the mushrooms (48.2\%) are poisonous.

We'll use ggplot2 to create some visualizations of various attributes
and how the relate to edibility.

Checking out the first two attributes:

\subsubsection{Cap Shape and CapSurface}\label{cap-shape-and-capsurface}

\begin{Shaded}
\begin{Highlighting}[]
\KeywordTok{ggplot}\NormalTok{(mushrooms, }\KeywordTok{aes}\NormalTok{(}\DataTypeTok{x =}\NormalTok{ cap.shape, }\DataTypeTok{y =}\NormalTok{ cap.surface, }\DataTypeTok{col =}\NormalTok{ class)) }\OperatorTok{+}\StringTok{ }
\StringTok{  }\KeywordTok{scale_color_manual}\NormalTok{(}\DataTypeTok{breaks =} \KeywordTok{c}\NormalTok{(}\StringTok{"edible"}\NormalTok{, }\StringTok{"poisonous"}\NormalTok{), }
                     \DataTypeTok{values =} \KeywordTok{c}\NormalTok{(}\StringTok{"green"}\NormalTok{, }\StringTok{"red"}\NormalTok{)) }\OperatorTok{+}
\StringTok{  }\KeywordTok{geom_jitter}\NormalTok{(}\DataTypeTok{alpha =} \FloatTok{0.7}\NormalTok{) }
\end{Highlighting}
\end{Shaded}

\includegraphics{MushroomsReport_files/figure-latex/unnamed-chunk-7-1.pdf}

We can see a couple of things in this plot: There is a very low
(one-digit) number of mushrooms that have cap surfaces with grooves or
conical cap shapes. All of those are poisonous Mushrooms with bell
shaped caps are mostly edible. Those with flat or convex cap shapes are
mostly poisonous.

Checking out the next attributes:

\subsubsection{Cap Color and Odor}\label{cap-color-and-odor}

\begin{Shaded}
\begin{Highlighting}[]
\KeywordTok{ggplot}\NormalTok{(mushrooms, }\KeywordTok{aes}\NormalTok{(}\DataTypeTok{x =}\NormalTok{ cap.color, }\DataTypeTok{y =}\NormalTok{ odor, }\DataTypeTok{col =}\NormalTok{ class)) }\OperatorTok{+}\StringTok{ }
\StringTok{  }\KeywordTok{scale_color_manual}\NormalTok{(}\DataTypeTok{breaks =} \KeywordTok{c}\NormalTok{(}\StringTok{"edible"}\NormalTok{, }\StringTok{"poisonous"}\NormalTok{), }
                     \DataTypeTok{values =} \KeywordTok{c}\NormalTok{(}\StringTok{"green"}\NormalTok{, }\StringTok{"red"}\NormalTok{)) }\OperatorTok{+}
\StringTok{  }\KeywordTok{geom_jitter}\NormalTok{(}\DataTypeTok{alpha =} \FloatTok{0.7}\NormalTok{) }
\end{Highlighting}
\end{Shaded}

\includegraphics{MushroomsReport_files/figure-latex/unnamed-chunk-8-1.pdf}

You definitely want to avoid any odor that is fishy, spicy, pungent,
musty, foul or creosote. Anise and almond seem to be very safe.
Mushrooms with no odor are mostly safe, but some are poisonous.

Checking out more attributes:

\subsubsection{Gill Color and Gill Size}\label{gill-color-and-gill-size}

\begin{Shaded}
\begin{Highlighting}[]
\KeywordTok{ggplot}\NormalTok{(mushrooms, }\KeywordTok{aes}\NormalTok{(}\DataTypeTok{x =}\NormalTok{ gill.color, }\DataTypeTok{y =}\NormalTok{ gill.size, }\DataTypeTok{col =}\NormalTok{ class)) }\OperatorTok{+}\StringTok{ }
\StringTok{  }\KeywordTok{scale_color_manual}\NormalTok{(}\DataTypeTok{breaks =} \KeywordTok{c}\NormalTok{(}\StringTok{"edible"}\NormalTok{, }\StringTok{"poisonous"}\NormalTok{), }
                     \DataTypeTok{values =} \KeywordTok{c}\NormalTok{(}\StringTok{"green"}\NormalTok{, }\StringTok{"red"}\NormalTok{)) }\OperatorTok{+}
\StringTok{  }\KeywordTok{geom_jitter}\NormalTok{(}\DataTypeTok{alpha =} \FloatTok{0.7}\NormalTok{) }
\end{Highlighting}
\end{Shaded}

\includegraphics{MushroomsReport_files/figure-latex/unnamed-chunk-9-1.pdf}

The Gill Colors red and orange are safe. Green is always poisonous. If
the Gill size is groad the following colors are safe too: black,brown,
orange, purple.

\subsubsection{Stark color above and below the
rind}\label{stark-color-above-and-below-the-rind}

\begin{Shaded}
\begin{Highlighting}[]
\KeywordTok{ggplot}\NormalTok{(mushrooms, }\KeywordTok{aes}\NormalTok{(}\DataTypeTok{x =}\NormalTok{ stalk.color.above.ring , }\DataTypeTok{y =}\NormalTok{ stalk.surface.above.ring, }\DataTypeTok{col =}\NormalTok{ class)) }\OperatorTok{+}\StringTok{ }
\StringTok{  }\KeywordTok{scale_color_manual}\NormalTok{(}\DataTypeTok{breaks =} \KeywordTok{c}\NormalTok{(}\StringTok{"edible"}\NormalTok{, }\StringTok{"poisonous"}\NormalTok{), }
                     \DataTypeTok{values =} \KeywordTok{c}\NormalTok{(}\StringTok{"green"}\NormalTok{, }\StringTok{"red"}\NormalTok{)) }\OperatorTok{+}
\StringTok{  }\KeywordTok{geom_jitter}\NormalTok{(}\DataTypeTok{alpha =} \FloatTok{0.7}\NormalTok{) }
\end{Highlighting}
\end{Shaded}

\includegraphics{MushroomsReport_files/figure-latex/unnamed-chunk-10-1.pdf}

We can see many combinations which are defintelly to avoid (buff/green,
brow) and a few that are safe (smooth + red/gray/pink)

Finally we take a look at:

\subsubsection{Population and spore print
color}\label{population-and-spore-print-color}

\begin{Shaded}
\begin{Highlighting}[]
\KeywordTok{ggplot}\NormalTok{(mushrooms, }\KeywordTok{aes}\NormalTok{(}\DataTypeTok{x =}\NormalTok{ spore.print.color , }\DataTypeTok{y =}\NormalTok{ population, }\DataTypeTok{col =}\NormalTok{ class)) }\OperatorTok{+}\StringTok{ }
\StringTok{  }\KeywordTok{scale_color_manual}\NormalTok{(}\DataTypeTok{breaks =} \KeywordTok{c}\NormalTok{(}\StringTok{"edible"}\NormalTok{, }\StringTok{"poisonous"}\NormalTok{), }
                     \DataTypeTok{values =} \KeywordTok{c}\NormalTok{(}\StringTok{"green"}\NormalTok{, }\StringTok{"red"}\NormalTok{)) }\OperatorTok{+}
\StringTok{  }\KeywordTok{geom_jitter}\NormalTok{(}\DataTypeTok{alpha =} \FloatTok{0.7}\NormalTok{) }
\end{Highlighting}
\end{Shaded}

\includegraphics{MushroomsReport_files/figure-latex/unnamed-chunk-11-1.pdf}

Again, we see several combinations which are safe, incl. any mushrooms
with a numerous population or buff, orange , purple, or yellow spore
prints.

To finalize our initial data exploration, let's look at three attributes
that seem to play a big role in prediting edibility:

\subsubsection{Cap Color, Sport Print Color and
Odor}\label{cap-color-sport-print-color-and-odor}

\begin{Shaded}
\begin{Highlighting}[]
\KeywordTok{ggplot}\NormalTok{(mushrooms, }\KeywordTok{aes}\NormalTok{(}\DataTypeTok{x =}\NormalTok{ mushrooms}\OperatorTok{$}\NormalTok{cap.color  , }\DataTypeTok{y =}\NormalTok{ mushrooms}\OperatorTok{$}\NormalTok{spore.print.color, }\DataTypeTok{col =}\NormalTok{ class)) }\OperatorTok{+}\StringTok{ }
\StringTok{  }\KeywordTok{scale_color_manual}\NormalTok{(}\DataTypeTok{breaks =} \KeywordTok{c}\NormalTok{(}\StringTok{"edible"}\NormalTok{, }\StringTok{"poisonous"}\NormalTok{), }
                     \DataTypeTok{values =} \KeywordTok{c}\NormalTok{(}\StringTok{"green"}\NormalTok{, }\StringTok{"red"}\NormalTok{)) }\OperatorTok{+}
\StringTok{  }\KeywordTok{geom_jitter}\NormalTok{(}\DataTypeTok{alpha =} \FloatTok{0.7}\NormalTok{) }\OperatorTok{+}\StringTok{ }\KeywordTok{facet_wrap}\NormalTok{(mushrooms}\OperatorTok{$}\NormalTok{odor) }\OperatorTok{+}
\StringTok{  }\KeywordTok{xlab}\NormalTok{(}\StringTok{"Cap Color"}\NormalTok{) }\OperatorTok{+}\StringTok{ }
\StringTok{  }\KeywordTok{ylab}\NormalTok{(}\StringTok{"Spore Print Color"}\NormalTok{) }\OperatorTok{+}
\StringTok{  }\KeywordTok{theme}\NormalTok{(}\DataTypeTok{axis.text.x =} \KeywordTok{element_text}\NormalTok{(}\DataTypeTok{angle =} \DecValTok{90}\NormalTok{))}
\end{Highlighting}
\end{Shaded}

\includegraphics{MushroomsReport_files/figure-latex/unnamed-chunk-12-1.pdf}

There are clearly pattern, so let's try to find a model to predict if a
mushroom is edible or not.

It looks like we might be able to make a good prediction using a
regression model, but we already used that in the previous MovieLens
exercise (\url{https://github.com/prangberg/harvard-data-science}) and
want to use a `real' machine learning model this time.

\subsection{Modeling Approach}\label{modeling-approach}

Split the data into training and a 10\% validation set:

\begin{Shaded}
\begin{Highlighting}[]
\KeywordTok{set.seed}\NormalTok{(}\DecValTok{1}\NormalTok{)}
\CommentTok{# Validation set will be 10% }
\NormalTok{test_index <-}
\StringTok{  }\KeywordTok{createDataPartition}\NormalTok{(}
    \DataTypeTok{y =}\NormalTok{ mushrooms}\OperatorTok{$}\NormalTok{class,}
    \DataTypeTok{times =} \DecValTok{1}\NormalTok{,}
    \DataTypeTok{p =} \FloatTok{0.1}\NormalTok{,}
    \DataTypeTok{list =} \OtherTok{FALSE}
\NormalTok{  )}
\NormalTok{edx <-}\StringTok{ }\NormalTok{mushrooms[}\OperatorTok{-}\NormalTok{test_index, ]}
\NormalTok{validation <-}\StringTok{ }\NormalTok{mushrooms[test_index, ]}
\end{Highlighting}
\end{Shaded}

\subsubsection{Random Forest}\label{random-forest}

We'll use the Random Forest algorithm

\begin{Shaded}
\begin{Highlighting}[]
\KeywordTok{set.seed}\NormalTok{(}\DecValTok{1}\NormalTok{)}
\NormalTok{model_rf <-}\StringTok{ }\KeywordTok{randomForest}\NormalTok{(class }\OperatorTok{~}\StringTok{ }\NormalTok{., }\DataTypeTok{ntree =} \DecValTok{30}\NormalTok{, }\DataTypeTok{data =}\NormalTok{ edx)}
\KeywordTok{plot}\NormalTok{(model_rf)}
\end{Highlighting}
\end{Shaded}

\includegraphics{MushroomsReport_files/figure-latex/unnamed-chunk-14-1.pdf}

The plot shows that above 15 trees, the error isn't decreasing anymore
and is very close to 0.

We use the model to create a prediction for our training data set.

\begin{Shaded}
\begin{Highlighting}[]
\NormalTok{edx}\OperatorTok{$}\NormalTok{predicted <-}\StringTok{ }\KeywordTok{predict}\NormalTok{(model_rf ,edx)}
\KeywordTok{confusionMatrix}\NormalTok{(}\DataTypeTok{data =}\NormalTok{ edx}\OperatorTok{$}\NormalTok{predicted, }\DataTypeTok{reference =}\NormalTok{ edx}\OperatorTok{$}\NormalTok{class , }
                \DataTypeTok{positive =} \StringTok{"edible"}\NormalTok{)}
\end{Highlighting}
\end{Shaded}

\begin{verbatim}
## Confusion Matrix and Statistics
## 
##            Reference
## Prediction  edible poisonous
##   edible      3787         0
##   poisonous      0      3524
##                                      
##                Accuracy : 1          
##                  95% CI : (0.9995, 1)
##     No Information Rate : 0.518      
##     P-Value [Acc > NIR] : < 2.2e-16  
##                                      
##                   Kappa : 1          
##  Mcnemar's Test P-Value : NA         
##                                      
##             Sensitivity : 1.000      
##             Specificity : 1.000      
##          Pos Pred Value : 1.000      
##          Neg Pred Value : 1.000      
##              Prevalence : 0.518      
##          Detection Rate : 0.518      
##    Detection Prevalence : 0.518      
##       Balanced Accuracy : 1.000      
##                                      
##        'Positive' Class : edible     
## 
\end{verbatim}

Using this model, there are no errors for the training set: This is a
perfect prediction of which mushrooms are edible or poisonous.

Let's take a quick look at which of the attributes of a mushroom play
the most important role in predicting its edibility:

\begin{Shaded}
\begin{Highlighting}[]
\NormalTok{var_imp <-}\KeywordTok{importance}\NormalTok{(model_rf) }\OperatorTok\StringTok{ }\KeywordTok{data.frame}\NormalTok{() }\OperatorTok\StringTok{ }
\StringTok{  }\KeywordTok{rownames_to_column}\NormalTok{(}\DataTypeTok{var =} \StringTok{"Variable"}\NormalTok{) }\OperatorTok\StringTok{ }
\StringTok{  }\KeywordTok{arrange}\NormalTok{(}\KeywordTok{desc}\NormalTok{(MeanDecreaseGini))}

\CommentTok{#Importance of attributes}
\KeywordTok{ggplot}\NormalTok{(var_imp, }\KeywordTok{aes}\NormalTok{(}\DataTypeTok{x=}\KeywordTok{reorder}\NormalTok{(Variable, MeanDecreaseGini), }\DataTypeTok{y=}\NormalTok{MeanDecreaseGini, }\DataTypeTok{fill=}\NormalTok{MeanDecreaseGini)) }\OperatorTok{+}
\StringTok{  }\KeywordTok{geom_bar}\NormalTok{(}\DataTypeTok{stat =} \StringTok{'identity'}\NormalTok{) }\OperatorTok{+}
\StringTok{  }\KeywordTok{geom_point}\NormalTok{() }\OperatorTok{+}
\StringTok{  }\KeywordTok{coord_flip}\NormalTok{() }\OperatorTok{+}
\StringTok{  }\KeywordTok{xlab}\NormalTok{(}\StringTok{"Factors"}\NormalTok{) }\OperatorTok{+}
\StringTok{  }\KeywordTok{ggtitle}\NormalTok{(}\StringTok{"Importance of Attributes"}\NormalTok{)}
\end{Highlighting}
\end{Shaded}

\includegraphics{MushroomsReport_files/figure-latex/unnamed-chunk-16-1.pdf}

A higher descrease in Gini means that a particular predictor variable
plays a greater role in partitioning the data into the defined classes.
The plot indicates that \textbf{Odor} is the most predicive variable in
determining edibility.

We can confirm the importance of odor when looking at this plot:

\begin{Shaded}
\begin{Highlighting}[]
\KeywordTok{ggplot}\NormalTok{(mushrooms, }\KeywordTok{aes}\NormalTok{(}\DataTypeTok{x =}\NormalTok{ odor, }\DataTypeTok{y =}\NormalTok{ class, }\DataTypeTok{col =}\NormalTok{ class)) }\OperatorTok{+}\StringTok{ }
\StringTok{  }\KeywordTok{scale_color_manual}\NormalTok{(}\DataTypeTok{breaks =} \KeywordTok{c}\NormalTok{(}\StringTok{"edible"}\NormalTok{, }\StringTok{"poisonous"}\NormalTok{), }
                     \DataTypeTok{values =} \KeywordTok{c}\NormalTok{(}\StringTok{"green"}\NormalTok{, }\StringTok{"red"}\NormalTok{))}\OperatorTok{+}\StringTok{ }\KeywordTok{geom_point}\NormalTok{(}\DataTypeTok{position=}\StringTok{'jitter'}\NormalTok{) }
\end{Highlighting}
\end{Shaded}

\includegraphics{MushroomsReport_files/figure-latex/unnamed-chunk-17-1.pdf}

Actually any mushroom that does have an odor (i.e.~not `none') can be
predicted: Almond or Anise are always safe. Creosote, foul, pungent,
spicy or fishy are always poisonous. Only those mushrooms without odor
cannot be predicted, even though it seems that most are edible.

-\textgreater{} So if there was only \emph{ONE} single attribute you can
consider when you find a mushroom, it should be odor.

Taking a quick look at the second most important factor \emph{Sport
Print Color}, in combination with odor:

\begin{Shaded}
\begin{Highlighting}[]
\KeywordTok{ggplot}\NormalTok{(mushrooms, }\KeywordTok{aes}\NormalTok{(}\DataTypeTok{x =}\NormalTok{ spore.print.color, }\DataTypeTok{y =}\NormalTok{ odor, }\DataTypeTok{col =}\NormalTok{ class)) }\OperatorTok{+}\StringTok{ }
\StringTok{  }\KeywordTok{scale_color_manual}\NormalTok{(}\DataTypeTok{breaks =} \KeywordTok{c}\NormalTok{(}\StringTok{"edible"}\NormalTok{, }\StringTok{"poisonous"}\NormalTok{), }
                     \DataTypeTok{values =} \KeywordTok{c}\NormalTok{(}\StringTok{"green"}\NormalTok{, }\StringTok{"red"}\NormalTok{)) }\OperatorTok{+}
\StringTok{  }\KeywordTok{geom_jitter}\NormalTok{(}\DataTypeTok{alpha =} \FloatTok{0.6}\NormalTok{) }\OperatorTok{+}
\StringTok{  }\KeywordTok{xlab}\NormalTok{(}\StringTok{"Sport Print Color"}\NormalTok{)}
\end{Highlighting}
\end{Shaded}

\includegraphics{MushroomsReport_files/figure-latex/unnamed-chunk-18-1.pdf}

This plot helps reduce the large uncertainty we had with odorless
(odor=none) mushrooms. We can eat any odorless mushroom exceopt those
with green or white spore print color. We could continue this exercise
and would probably come up with a very precise regression model, but
let's focus on our Random Tree prediction.

\subsubsection{Validation}\label{validation}

Let's apply the Random Forest model to our validation set and see how
good the predictions are:

\begin{Shaded}
\begin{Highlighting}[]
\NormalTok{validation}\OperatorTok{$}\NormalTok{predicted <-}\StringTok{ }\KeywordTok{predict}\NormalTok{(model_rf ,validation)}

\KeywordTok{confusionMatrix}\NormalTok{(}\DataTypeTok{data =}\NormalTok{ validation}\OperatorTok{$}\NormalTok{predicted, }\DataTypeTok{reference =}\NormalTok{ validation}\OperatorTok{$}\NormalTok{class , }\DataTypeTok{positive =} \StringTok{"edible"}\NormalTok{)}
\end{Highlighting}
\end{Shaded}

\begin{verbatim}
## Confusion Matrix and Statistics
## 
##            Reference
## Prediction  edible poisonous
##   edible       421         0
##   poisonous      0       392
##                                      
##                Accuracy : 1          
##                  95% CI : (0.9955, 1)
##     No Information Rate : 0.5178     
##     P-Value [Acc > NIR] : < 2.2e-16  
##                                      
##                   Kappa : 1          
##  Mcnemar's Test P-Value : NA         
##                                      
##             Sensitivity : 1.0000     
##             Specificity : 1.0000     
##          Pos Pred Value : 1.0000     
##          Neg Pred Value : 1.0000     
##              Prevalence : 0.5178     
##          Detection Rate : 0.5178     
##    Detection Prevalence : 0.5178     
##       Balanced Accuracy : 1.0000     
##                                      
##        'Positive' Class : edible     
## 
\end{verbatim}

This is a perfect prediction - Accuracy, Sensitivity and Specificity are
1.00.

\section{Results}\label{results}

The Model using Random Forests gives a 100\% accurate prediction if a
mushroom is edible or not - based on its attributes. Above 15 trees, the
error isn't decreasing anymore and is equal to 0.

\section{Conclusion}\label{conclusion}

Odor is the most predicive variable in determining edibility. Almond or
Anise are always safe. Creosote, foul, pungent, spicy or fishy are
always poisonous. So if there was only ONE single attribute you can
consider when you find a mushroom, it should be odor.

Using the other attributes in a Random Forest model, we can predict with
100\% accuracy, if a mushroom is edible or not.

It would be interesting to apply this model to a larger dataset which
containts more than 23 species of gilled mushrooms in the Agaricus and
Lepiota Family. This might require a higher number of Random Trees or a
better fitted model.


\end{document}
